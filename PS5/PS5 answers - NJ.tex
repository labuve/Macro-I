\documentclass[]{article}
\usepackage{amsmath, amsfonts}
\usepackage[inline]{enumitem}
\usepackage{fancyhdr}
\usepackage{geometry}
\usepackage{cancel}
\usepackage{graphicx}
\usepackage{color}
\usepackage{subcaption}
\usepackage{cleveref}
\usepackage{MnSymbol,wasysym}
\usepackage{titlesec}

%opening
\title{Problem Set V \\ \large Macroeconomics I}
\author{Nurfatima Jandarova}
\date{\today}
\pagestyle{fancy}

\lhead{Macroeconomics I, Problem Set V}
\rhead{Nurfatima Jandarova}
\renewcommand{\headrulewidth}{0.4pt}
\fancyheadoffset{1 cm}

\geometry{a4paper, left=20mm, top=20mm, bottom = 20mm, headheight=20mm}

\sloppy
\definecolor{lightgray}{gray}{0.5}
\setlength{\parindent}{0pt}

\renewcommand{\thesubsection}{Exercise \arabic{subsection}}
\renewcommand{\thesubsubsection}{\textbf{(\alph{subsubsection})}}
\titleformat{\subsubsection}[runin]
{\normalfont\normalsize}{\thesubsubsection}{1em}{}


\begin{document}

\maketitle

\subsection{Corners again}

\subsubsection{}
\begin{itemize}
	\item time is discrete and infinite
	\item Define $s_t\in W$ the realization of a stochastic event that defines the wage rate. Hence, history of events is denoted as $s^t$. Define as well the associated transition matrix, $P(s_{t+1}|s_t)$ and the unconditional probability of each history $s^t$ at time $t$, $\pi_t(s^t)$.
	\item Household values $\mathbb{E}_0\sum\limits_{t = 0}^\infty\beta^t[\ln(c_t(s^t)) - (1 - l_t(s^t))] = \sum\limits_{t = 0}^\infty\sum\limits_{s^t}\beta^t\pi_t(s^t)[\ln(c_t(s^t)) - n_t(s^t)], \quad\beta\in(0, 1)$. Notice that period utility is a  $\mathcal{C}^2$, increasing in both $c_t(s^t)$ and $l_t(s^t)$, and a concave function. Furthermore, period utility function satisfies Inada condition with respect to consumption, but not with respect to leisure: $\lim\limits_{c\to0}u_c(c, l) = \lim\limits_{c\to0}\frac{1}{c} = \infty, \forall c$, $\lim\limits_{l\to0}u_l(c, l) = \lim\limits_{l\to0}1 = 1, \forall l$.
	\item At time $t$ and history $s^t$, a consumer is paid $w_t(s_t)$. Notice, that wage is said to follow Markov process, i.e., is history-independent. Therefore, the allocations in the economy are as well history-independent.
	\item Although not directly specified by the problem, I assume sequential trade market structure. Define $a_{t+1}(s^t, s_{t+1})$ the amount of claims on time $t+1$, history $s^{t+1}$ consumption bought at time $t$, history $s^t$.
	\item Consumer has to choose allocations $\{c_t(s^t), n_t(s^t)\}_{t = 0}^\infty$ and asset positions $a_{t+1}(s^t, s_{t+1})$ to
	\begin{equation}
		\begin{split}
			\max\limits_{\{c_t(s_t), n_t(s_t), a_{t+1}(s_t, s_{t+1})\}_{t = 0}^\infty}&\sum\limits_{t = 0}^\infty\beta^t\sum\limits_{s^t}\pi_t(s^t)[\ln(c_t(s_t)) - n_t(s_t)]\text{ s.t. }\\
			&c_t(s_t) + \sum\limits_{s_{t+1}|s^t}\frac{a_{t+1}(s_t, s_{t+1})}{1+r} = w_t(s_t)n_t(s_t) + a_t(s_t), \qquad\forall t, \forall s^t\nonumber\\
			&a_{t+1}(s_t, s_{t+1}) \geq B, \qquad \forall t, \forall s_t, \forall s_{t+1}\\
			&a_0(s_0) = 0
		\end{split}
	\end{equation}
\end{itemize}
Hence, the problem could be written recursively as 
\begin{equation}
	\begin{split}
		V(a, s) = \max\limits_{c, n, a'}\{\ln(c) - n + \beta\sum\limits_{s'\in W}P(s'|s)V(a'(s'), s')\}\text{ s.t. }&c + \sum\limits_{s'\in W}\frac{a'(s')}{1 + r} = w(s)n + a \nonumber \\
		& a'(s')\geq B,  \qquad\forall s'\in W
		\intertext{or}
		V(a, s) = \max\limits_{n, a'}\begin{Bmatrix}\ln\begin{pmatrix}w(s)n + a - \sum\limits_{s'\in W}\frac{a'(s')}{1 + r}\end{pmatrix} - n + \beta\sum\limits_{s'\in W}P(s'|s)V(a'(s'), s')\end{Bmatrix}\text{ s.t. } & a'(s')\geq B,  \quad\forall s'\in W
	\end{split}
\end{equation}

\subsubsection{}
If $B$ is the natural debt limit, then the borrowing constraint will never bind. Suppose it does, which then implies that the household will be bound to consume 0 for the rest of infinite life. However, consumer's period utility satisfies Inada condition with respect to consumption, i.e., even a slightest deviation form zero constitutes a vast utility improvement. Hence, at equilibrium consumer will never want to have zero consumption.

\subsubsection{}
Policy functions for the household are given by $\sigma^c(a, s), \sigma^n(a, s), \sigma^a(a, s, s')$. Then, the FOCs are:
\begin{equation}
	\begin{split}
		\frac{1}{\sigma^c(a, s)}& = \mu \\ \nonumber
		-1 + \mu w(s) = -1 + \frac{w(s)}{\sigma^c(a, s)}& \\
		\begin{matrix}
		&\frac{\mu}{1 + r} = \beta P(s'|s)\frac{\partial V(a'(s'), s')}{\partial a'(s')}\\
		&\frac{\partial V(a, s)}{\partial a} = \mu
		\end{matrix}& \Longrightarrow \frac{1}{1 + r} = \beta P(s'|s) \frac{\sigma^c(a, s)}{\sigma^c(\sigma^a(a, s, s'), s')}
	\end{split}
\end{equation}


\subsection{Irreversible capital accumulation}

\subsubsection{}
The problem could be written recursively
\begin{equation}
	\begin{split}
		V(a, k)& = \max\limits_{c, k'}\ln(c) + \beta\mathbb{E}_{a'|a}V(a', k')\\\nonumber
		\text{ s.t. }& c + k' = ak^\alpha + (1 - \delta)k \\
		& k' - (1 - \delta)k\geq 0
		\intertext{FOCs:}
		& \begin{cases}
			\frac{1}{c} = \mu \\
			\beta\mathbb{E}_{a'|a}\frac{\partial V(a', k')}{\partial k'} - \mu + \lambda = 0\\
			\min\{\lambda, k' - (1 - \delta)k\} = 0
		\end{cases}
		\intertext{Envelope condition}
		\frac{\partial V(a, k)}{\partial k}& = \mu(a\alpha k^{\alpha - 1} + 1 - \delta) - \lambda(1 - \delta)
	\end{split}
\end{equation}

\subsubsection{}

Combining the FOCs and the envelope condition we get
\begin{equation}
	\underbrace{\beta\mathbb{E}_{a'|a}\begin{bmatrix}\frac{a'\alpha(k')^{\alpha - 1} + 1 - \delta}{c'}\end{bmatrix}}_{\substack{\text{discounted expected marginal} \\ \text{benefit of consumption tomorrow}}} + \underbrace{\lambda - \beta\lambda'(1 - \delta)}_{(*)} = \underbrace{\frac{1}{c}}_{\substack{\text{marginal benefit of}\\\text{consumption today}}}
	\label{eq:ex2finfoc}
\end{equation}
(*) It seems to me that this expression gives an intertemporal value of relaxing irreversibility constraint today versus tomorrow. In a sense, due to depreciation of capital it is "easier" to relax irreversibility constraint in the future than today, but subject to discounting as it takes place in the future. 

Suppose that $k' > (1 - \delta)k \Rightarrow \lambda = 0$. Then, \eqref{eq:ex2finfoc} turns into an Euler equation in standard RBC model:
\begin{equation}
	\beta\mathbb{E}_{a'|a}\begin{bmatrix}\frac{a'\alpha(k')^{\alpha - 1} + 1 - \delta}{c'}\end{bmatrix} = \frac{1}{c} \nonumber
\end{equation}
Now it is easier to see that if $\lambda > 0 \Rightarrow k' = (1 - \delta)k$, then an agent ideally would want to start eating up the capital stock, but cannot. Hence, an agent ends up with higher $k'$ and lower $c$ compared to the standard case with reversible investment. This means that
\begin{equation}
	\beta\mathbb{E}_{a'|a}\begin{bmatrix}\frac{a'\alpha(k')^{\alpha - 1} + 1 - \delta}{c'}\end{bmatrix} < \frac{1}{c} \nonumber
\end{equation}
, i.e., again consumer would have been better off consuming more today, but is bound to postpone consumption to the future period.

\subsubsection{}
As long as the depreciation rate of capital $\delta > 0$, irreversibility constraint does not affect the non-stochastic steady-state of capital $k^*$.

\underline{Algebraically:}
\begin{enumerate}
	\item Remove shocks: $a = \mathbb{E}(a) = 1$
	\item $k$ such that $k = k' = k^*$. This means that $k^* - (1 - \delta)k^* > 0$ if $\delta > 0$. This also implies that $\Rightarrow c = c' = c^* \Longrightarrow$
	\begin{equation}
		\begin{split}
			\frac{1}{\cancel{c^*}}& = \beta\frac{\alpha(k^*)^{\alpha - 1} + 1 - \delta}{\cancel{c^*}} \nonumber \\
			k^*& = \begin{bmatrix}\frac{1}{\alpha}\begin{pmatrix}\frac{1}{\beta} - (1 - \delta)\end{pmatrix}\end{bmatrix}^{\frac{1}{\alpha - 1}}
		\end{split}
	\end{equation}
	i.e., same as the steady-state capital in the standard model.
\end{enumerate}

\underline{Intuitively:}
Steady-state capital is defined as the level of capital constant over time. To keep it constant over time in presence of capital depreciation, investment needs to be positive. Thus, irreversibility of investment has no impact on the steady-state level capital.

\subsubsection{}

As in standard case, value function is increasing in the productivity level. Higher productivity level allows to produce more for a given level of capital, and hence allows agents to enjoy higher life-time utility. At the same time, higher productivity level also allows agents to both consume and save more. Therefore, the intersection of capital decision rule with 45$^\circ$ line is higher for higher productivity level, i.e., higher productivity level allows agents to accumulate capital longer.

\begin{figure}[h]
	\centering
	\begin{subfigure}{0.45\textwidth}
		\centering
		\includegraphics[width = 0.95\linewidth]{ex2kpolicy.eps}
	\end{subfigure} ~
	\begin{subfigure}{0.45\textwidth}
		\centering
		\includegraphics[width = 0.95\linewidth]{ex2value.eps}
	\end{subfigure}
\end{figure}

Simulation of the model for 2000 periods starting at the steady-state level shows that capital path is roughly levelled out over time with fluctuations due to productivity shocks. Simulation of the investment path depicts that irreversibility constraint never binds. The reason for this could be that consumer wants to smooth consumption over time. Hence, it is never optimal for a consumer to start eating up capital stock because it increases consumption today at the cost of permanently shifting consumption downwards in the (infinite) future. 

\begin{figure}[h]
	\centering
	\begin{subfigure}{0.45\textwidth}
		\centering
		\includegraphics[width = 0.95\linewidth]{ex2kpath.eps}
	\end{subfigure} ~
	\begin{subfigure}{0.45\textwidth}
		\centering
		\includegraphics[width = 0.95\linewidth]{ex2ipath.eps}
	\end{subfigure}
\end{figure}

\subsection{Shocks to depreciation of capital}

\subsubsection{}

Let $\delta_t(s_t)$ denote the realization of the stochastic shock to the depreciation rate at time $t$. Since we are given that the depreciation rate follows a stationary Markov process, the allocations are history-independent. Combining the law of motion and the resource constraint we get $c_t(s_t) + K_{t+1}(s_t) = f(K_t(s_{t-1})) + (1 - \delta_t(s_t))K_t(s_{t-1})$. Thus, the planner's problem is to

\begin{equation}
	\begin{split}
		\max\limits_{\{c_t(s_t), K_{t+1}(s_t)\}_{t = 0}^\infty}\sum\limits_{t = 0}^\infty\sum\limits_{s^t}\beta^t\pi_t(s^t)\frac{(c_t(s_t))^{1 - \sigma} + 1}{1 - \sigma}\text{ s.t. }&c_t(s_t) + K_{t+1}(s_t) = f(K_t(s_{t-1})) + (1 - \delta_t(s_t))K_t(s_{t-1}), \quad\forall t, \forall s^t \nonumber \\
		&c_t(s_t)\geq 0, \quad\forall t, \forall s_t \\
		&K_0 \text{ given}
	\end{split}
\end{equation}

The planner's problem could be written recursively as
\begin{equation}
	\begin{split}
		V(K, s)& = \max\limits_{K'}\frac{(f(K) + (1-\delta(s))K - K')^{1 - \sigma} + 1}{1 - \sigma} + \beta\sum\limits_{s'|s}P(s'|s)V(K', s') \nonumber
		\intertext{FOC:}
		&\beta\sum\limits_{s'|s}P(s'|s)\frac{\partial V(K', s')}{\partial K'} = (f(K) + (1-\delta(s))K - K')^{-\sigma} \\
		\text{Envelope condition: }&\frac{\partial V(K, s)}{\partial K} = (f(K) + (1-\delta(s))K - K')^{-\sigma}(f'(K) + 1-\delta(s)) \Longrightarrow\\
		&\beta\sum\limits_{s'|s}P(s'|s)(f(K') + (1-\delta(s'))K' - K'')^{-\sigma}(f'(K') + 1-\delta(s')) = (f(K) + (1-\delta(s))K - K')^{-\sigma}\\
		&\underbrace{\beta\sum\limits_{s'|s}P(s'|s)(c')^{-\sigma}(f'(K') + 1-\delta(s'))}_{\substack{\text{discounted expected marginal benefit}\\\text{of consumption tomorrow}}} = \underbrace{c^{-\sigma}}_{\substack{\text{marginal benefit of}\\\text{consumption today}}}
	\end{split}
\end{equation}

\subsubsection{}

Define $q_t^0(s^t)$ as the price at time $t = 0$ of a claim on consumption at time $t$ and state $s_t$ (recall that allocations and prices in this problem are history-independent). Household seeks to maximize expected discounted lifetime utility subject to the budget constraint at given prices:
\begin{equation}
	\begin{split}
		\max\limits_{\{\{c_t(s_t)\}_{s_t}\}_{t = 0}^\infty}\sum\limits_{t = 0}^\infty\sum\limits_{s^t}\beta^t\pi_t(s^t)\frac{(c_t(s_t))^{1 - \sigma} + 1}{1 - \sigma}\text{ s.t. }\sum\limits_{t = 0}^\infty\sum\limits_{s^t} q_t^0(s_t)c_t(s_t) = \sum\limits_{t = 0}^\infty\sum\limits_{s^t}w_t^0(s_t) + p_{k_0}k_0 \nonumber
	\end{split}
\end{equation}

Type I (production) firm chooses capital and labour force to maximize profit:
\begin{equation}
	\begin{split}
		\max\limits_{\{\{k_t^I(s_t), n_t^I(s_t)\}_{s_t}\}_{t = 0}^\infty}\sum\limits_{t = 0}^\infty\sum\limits_{s_t}q_t^0(s_t)F(k_t^I(s_t), n_t^I(s_t)) - w_t^0(s_t)n_t^I(s_t) - r_t^0(s_t)k_t^I(s_t) \nonumber
	\end{split}
\end{equation}

Type II (investment) firm decides how much capital to rent out to the production firm and how much of initial capital to buy from the consumer. Hence, investment firm's problem:
\begin{equation}
	\begin{split}
		\max\limits_{\{\{k_0^{II}, k_{t+1}^{II}(s_t)\}_{s_t}\}_{t = 0}^\infty} -p_{k_0}k_0^{II} + \sum\limits_{t = 0}^\infty\sum\limits_{s_t}\{r_t^0(s_t)k_t^{II}(s_{t-1}) - q_t^0(s_t)[k_{t+1}^{II}(s_t) - (1 - \delta(s_t))k_t^{II}(s_{t-1})]\} \nonumber
	\end{split}
\end{equation}

An Arrow-Debreu competitive equilibrium is a sequence of prices $\{\{q_t^0(s_t), w_t^0(s_t), r_t^0(s_t), p_{k_0}\}_{s_t}\}_{t = 0}^\infty$ and allocations $\{\{c_t(s_t), k_t^I(s_t), n_t^I(s_t), k_{t+1}^{II}(s_t)\}\}$ such that
\begin{enumerate}[label=\roman*)]
	\item the allocations solve the maximization problems of households, Type I and Type II firms at a given $k_0$ and taking prices as given;
	\item prices are such that markets clear:
	\begin{equation}
		\begin{split}
			n_t^I(s_t)& = 1 \\ \nonumber
			k_t^I(s_t)& = k_t^{II}(s_{t-1}) \equiv k_t \\
			f(k_t) =: F(k_t, 1)& = c_t(s_t) + k_{t+1} - (1 - \delta(s_t))k_t
		\end{split}
	\end{equation}
\end{enumerate}
Since preferences are locally nonsatiated, markets are complete and there is a free disposal of output, we can apply First fundamental welfare theorem, which says that a competitive equilibrium allocation is also a Pareto efficient allocation.

\subsubsection{}
We forgo the $s$ notation and let the uncertainty to be captured by realization of $\delta_t\in\Delta$ with transition probabilities $P(\delta'|\delta)$. The aggregate state of the economy is given by $X \equiv (\delta, K)\in\mathcal{X}\equiv\Delta\times\mathbb{R}_+$. Define as well perceived law of motion for the aggregate state $X' = \hat{\Pi}(X'|X)$, which is made up of the perceived law of motion for the aggregate capital $K' = G(X)$ and a belief about evolution of the depreciation rate $\hat{P}(\delta'|\delta)$. We also assume there's a sequential trade of Arrow securities with a price today at aggregate state $X$ of a claim on consumption tomorrow when $\delta'$ realizes, $Q:\mathcal{X}\times\Delta\to\mathbb{R}, (X, \delta')\mapsto Q(\delta'|X)$. Let $a$ stand for the amount of Arrow securities at the current state $X$. Rental rate of capital and wage rate are defined as follows: $r: \mathcal{X}\to\mathbb{R}, X\mapsto r(X)$ and $w:\mathcal{X}\to\mathbb{R}, X\mapsto w(X)$, respectively.

Then, household's problem could be written recursively as
\begin{equation}
	\begin{split}
		V(a, X)& = \max\limits_{c, a'}\frac{c^{1 - \sigma} + 1}{1 - \sigma} + \beta\sum\limits_{\delta'\in\Delta}\hat{P}(\delta'|\delta)V(a'(\delta'), (\delta', G(X)))\text{ s.t. }c + \sum\limits_{\delta'\in\Delta}Q(\delta'|X)a'(\delta') = w(X) + a\nonumber\\
		\text{FOCs: }&(\phi^c(a, X))^{-\sigma} = \mu \\
		&\beta\hat{P}(\delta'|\delta)\frac{\partial V(\phi^{a'}(a, X, \delta'), (\delta', G(X)))}{\partial a'} = \mu Q(\delta'|X) \\
		\text{Envelope condition: }&\frac{\partial V(a, X)}{\partial a} = \mu \Longrightarrow \\
		& \beta\hat{P}(\delta'|\delta)(\phi^c(\phi^{a'}(a, X, \delta'), (\delta', G(X))))^{-\sigma} = (\phi^c(a, X))^{-\sigma} Q(\delta'|X)
	\end{split}
\end{equation}

Type I firm's problem:
\begin{equation}
	\begin{split}
		&\max_{k^I, n^I} F(k^I, n^I) - w(X)n^I - r(X)k^I \nonumber \\
		\text{FOCs: }&F_k(k^I, n^I) = r(X) \\
		&F_n(k^I, n^I) = w(X)
	\end{split}
\end{equation}

Type II firm's problem:
\begin{equation}
	\begin{split}
		\max\limits_{k^{II\prime}} &-k^{II\prime} + \sum\limits_{\delta^{\prime}\in\Delta}Q(\delta^{\prime}|X)\begin{bmatrix}r((\delta^{\prime}, G(X)))k^{II\prime} + (1 - \delta)k^{II\prime}\end{bmatrix} \nonumber \\
		&1 = \sum\limits_{\delta^{\prime}\in\Delta}Q(\delta^{\prime}|X)\begin{bmatrix}r((\delta^{\prime}, G(X))) + 1 - \delta\end{bmatrix}
	\end{split}
\end{equation}

Hence, a recursive competitive equilibrium is a price system $r(X), w(X), Q(\delta'|X)$, perceived laws of motion $G(X), \hat{P}(\delta'|\delta)$, household's value function $V(a, X)$ and policy functions $\phi^c(a, X), \phi^{a'}(a, X, \delta'), k^I(X), n^I(X), k^{II}(X)$ such that for a given $K_0$:
\begin{enumerate}[label=\roman*)]
	\item policy functions solve respective optimization problems of households, production and investment firms described above, taking prices and perceived laws of motion as given
	\item markets clear
	\begin{equation}
		\begin{split}
			k^I& = k^{II} = K \\ \nonumber
			n^I& = 1 \\
			c + K'& = f(K) + (1 - \delta)K \\
			a& = (r(X) + 1 - \delta)K
		\end{split}
	\end{equation}
	\item rational expectations
	\begin{equation}
		\begin{split}
		\hat{P}(\delta'|\delta)& = P(\delta'|\delta) \\ \nonumber
		G(X)& = \mathcal{G}(X)
		\end{split}
	\end{equation}
	where $\mathcal{G}(X)$ is found by substituting asset/capital market clearing into budget constraint of the household, FOCs of firms' problems and using Euler's theorem
	\begin{equation}
		\begin{split}
			c + K'\cancelto{1}{\sum\limits_{\delta'\in\Delta}Q(\delta'|X)(r(\delta', G(X)) + 1 - \delta')}& = w(X) + r(X)K + (1 - \delta)K \\ \nonumber
			c + K'& = F(K, 1) + (1 - \delta)K \\
			c + K'& = f(K) + (1 - \delta)K \\
			K'& = f(K) + (1 - \delta)K - \phi^c((r(X)K + (1 - \delta)K, X) \equiv \mathcal{G}(X)\\
		\end{split}
	\end{equation}
\end{enumerate}

%\subsection{Endogenous labour supply (old)}
%
%\subsubsection{}
%The planner's problem sequentially:
%\begin{equation}
%	\begin{split}
%		\mathcal{L}& = \sum\limits_{t = 0}^T\beta^t\begin{Bmatrix}\ln(c_t) - \ln(n_t) + \mu_t\begin{pmatrix}n_t^\alpha k_t^{1 - \alpha} + (1 - \delta)k_t - k_{t+1} - c_t\end{pmatrix}\end{Bmatrix} \nonumber \\
%		\intertext{FOCs:}
%		&\begin{cases}
%			\frac{\partial\mathcal{L}}{\partial c_t}& = \beta^t\begin{Bmatrix}\frac{1}{c_t} - \mu_t\end{Bmatrix} = 0, \qquad
%			\forall t\leq T - 1 \\
%			\frac{\partial\mathcal{L}}{\partial n_t}& = \beta^t\begin{Bmatrix}-\frac{1}{n_t} + \mu_t\alpha n_t^{\alpha - 1}k_t^{1 - \alpha}\end{Bmatrix} = 0, \qquad\forall t\leq T - 1 \\
%			\frac{\partial\mathcal{L}}{\partial k_{t+1}}& = -\beta^t\mu_t + \beta^{t+1}\mu_{t+1}\begin{pmatrix}(1 - \alpha)n_{t+1}^\alpha k_{t+1}^{ - \alpha} + 1 - \delta\end{pmatrix} = 0, \qquad\forall t\leq T - 1 \\
%			\frac{\partial\mathcal{L}}{\partial \mu_t}& = \beta^t\begin{pmatrix}n_t^\alpha k_t^{1 - \alpha} + (1 - \delta)k_t - k_{t+1} - c_t\end{pmatrix} = 0, \qquad\forall t\leq T-1
%		\end{cases} \Rightarrow 
%		\begin{cases}
%			\frac{1}{c_t}& = \mu_t \\
%			c_t& =  \alpha n_t^\alpha k_t^{1 - \alpha} \\
%			\frac{1}{c_t}& = \beta\frac{1}{c_{t+1}}\begin{pmatrix}(1 - \alpha) \begin{pmatrix}\frac{k_{t+1}}{n_{t+1}}\end{pmatrix}^{-\alpha} + 1 - \delta\end{pmatrix} \\
%			c_t + k_{t+1}& = n_t^\alpha k_t^{1 - \alpha} + (1 - \delta)k_t
%		\end{cases}
%		\intertext{Using the intratemporal FOC and resource constraint we get the following}
%		&\begin{cases}
%			c_t =  \alpha n_t^\alpha k_t^{1 - \alpha} \\
%			(1 - \alpha)n_t^\alpha k_t^{1 - \alpha} = k_{t + 1} - (1 - \delta)k_t
%		\end{cases} \Rightarrow \begin{cases}
%			c_t =  \alpha n_t^\alpha k_t^{1 - \alpha} \\
%			n_t  = \begin{bmatrix}
%			\frac{k_{t + 1} - (1 - \delta)k_t}{(1 - \alpha)k_t^{1 - \alpha}}
%			\end{bmatrix}^{\frac{1}{\alpha}}
%		\end{cases} \Rightarrow \begin{cases}
%			c_t = \frac{\alpha}{1 - \alpha}\begin{bmatrix}k_{t + 1} - (1 - \delta)k_t\end{bmatrix} \\
%			n_t  = \begin{bmatrix}
%			\frac{k_{t + 1} - (1 - \delta)k_t}{(1 - \alpha)k_t^{1 - \alpha}}
%			\end{bmatrix}^{\frac{1}{\alpha}}
%		\end{cases}
%		\intertext{Hence,}
%		&\begin{cases}
%			\tilde{c}(k_t, k_{t+1})& = \frac{\alpha}{1 - \alpha}\begin{bmatrix}k_{t + 1} - (1 - \delta)k_t\end{bmatrix}, \quad\forall t\leq T - 1 \\
%			\tilde{n}(k_t, k_{t+1})& = \begin{bmatrix}
%			\frac{k_{t + 1} - (1 - \delta)k_t}{(1 - \alpha)k_t^{1 - \alpha}}
%			\end{bmatrix}^{\frac{1}{\alpha}}, \quad\forall t\leq T - 1
%		\end{cases}
%	\end{split}
%\end{equation}
%
%Since the utility function is strictly increasing in $c_t$ nd decreasing in $n_t$, in the terminal period $t = T$, the following conditions hold:
%\begin{equation}
%	\begin{split}
%		k_{T + 1}& = 0 \\
%		n_T& = 0 \\
%		c_T& = (1 - \delta)k_T
%	\end{split}\nonumber
%\end{equation}
%
%%To write the planner's problem recursively, define the constrain sets for consumption, labour and capital: $n_t\in[0, 1]$, $c_{t}\in B(k_t):=[0, k_t^{1 - \alpha} + (1 - \delta)k_t]$, and $k_{t+1}\in\Gamma(k_t):=[0, k_t^{1 - \alpha} + (1 - \delta)k_t]$.
%
%The planner's problem recursively:
%
%\begin{equation}
%	\begin{split}
%		V_{s}(k_{T - s})& = \max\limits_{n_{T - s}, c_{T - s}}\ln(c_{T - s}) - \ln(n_{T - s}) + \beta V_{s-1}(n_{T - s}^\alpha k_{T - s}^{1 - \alpha} + (1 - \delta)k_{T - s} - c_{T - s}) \nonumber \\
%		\intertext{FOCs:}
%		&\begin{cases}
%			\frac{1}{c_t}& = \beta V_{s - 1}'(n_t^\alpha k_t^{1 - \alpha} + (1 - \delta)k_t - c_t), \qquad\forall t\leq T - 1 \\
%			\frac{1}{n_t}& = \beta\alpha n_t^{\alpha - 1}k_t^{1 - \alpha}V_{s - 1}'(n_t^\alpha k_t^{1 - \alpha} + (1 - \delta)k_t - c_t), \qquad\forall t\leq T - 1 \\
%		\end{cases}	
%	\end{split}
%\end{equation}
%
%To proceed further, one needs to know the functional form of the value functions at each point in time. It seems to me that I'd need to use value function iteration or guess and verify to obtain functional forms of the value functions and optimal decision rules. Envelope condition does not apply because in a finite horizon, value functions do differ across time. So, for the numerical part I continue using first-order conditions derived from sequential problem as both problems should yield same result.
%
%\subsubsection{}
%
%Using the FOC in the previous part, numerical solution yields the following paths for capital, labour and consumption. Notice that optimal choice of capital for the next period is constant over time until it drops to zero when the world ends. Unlike the very simple finite-horizon growth model considered in Problem Set 1, social planner chooses the keep the stock of capital constant because \begin{enumerate*}[label=\roman*)]\item a unit of labour supplied results in higher disutility to the agent than an increase in the capital stock, and \item a lower supply of labour will result in gradual decrease of capital stock and a lower consumption plan for an agent \end{enumerate*}. Using results from the previous part, one can show that $k_{t+1} = k_t \Longrightarrow n_t = k_t\begin{pmatrix}\frac{\delta}{1 - \alpha}\end{pmatrix}^{\frac{1}{\alpha}} \approx 0.04$ (see \Cref{fig:ex4npath}). This allows an agent to enjoy a constant consumption plan for almost an entire life, except the terminal period, when he/she consumes everything that is left before the world ends (see \Cref{fig:ex4cpath}).
%
%\begin{figure}[h]
%	\centering
%	\begin{subfigure}{0.5\textwidth}
%		\centering
%		\includegraphics[width = 0.95\linewidth]{kpath.eps}
%		\caption{Optimal capital path starting at $k_0 = 2$}
%		\label{fig:ex4kpath}
%	\end{subfigure}~
%	\begin{subfigure}{0.5\textwidth}
%		\centering
%		\includegraphics[width = 0.95\linewidth]{npath.eps}
%		\caption{Optimal path for labour with $k_0 = 2$}
%		\label{fig:ex4npath}
%	\end{subfigure}
%	\begin{subfigure}{0.5\textwidth}
%		\centering
%		\includegraphics[width = 0.95\linewidth]{cpath.eps}
%		\caption{Optimal consumption path with $k_0 = 2$}
%		\label{fig:ex4cpath}
%	\end{subfigure}
%	\caption{Numerical solution to optimal policy rules}
%	\label{fig:ex4}
%\end{figure}

\end{document}